The term project for T\+T\+K4155 Embedded and Industrial Computer Systems Design for group 20 2016.

Created by Sondre Wangenstein Baugstø, Johan Lofstad and Sondre Vincent Russvoll.

\subsection*{File structure}

The library is shared between the two nodes, and can be found in the lib folder. Every class has its own folder, with its own .h and .cpp file.

If a module is specific to one chip, for example the A\+Tmega162, we have included a {\ttfamily \#ifdef \+\_\+\+\_\+\+A\+V\+R\+\_\+\+A\+Tmega162\+\_\+\+\_\+} in the header file. This define is set by the compiler during the build process.

The node specific code is located in the node(1/2) / src. The {\itshape init.\+cpp} file is ran each time the M\+CU starts, and {\itshape main.\+cpp} is the main file. The {\itshape state\+\_\+machine.\+cpp} is the state machine implementation.

\subsubsection*{State machine}

Each of the nodes runs on a state machine. The implementation of said state machine can be found in {\itshape lib/state\+\_\+machine}. Under {\itshape node(1/2) / src / state\+\_\+machine.\+cpp} we have specified and implemented the states of the node.

There is one function that runs when a state is entered, one that loops until the state is left, and one that runs when we leave the state. Transitions between states is done with {\ttfamily fsm-\/$>$Transition(\+S\+T\+A\+T\+E\+\_\+\+N\+E\+X\+T, 0)}

{\bfseries Advantages of using this F\+SM}

There are several advantages of using this kind of F\+SM. Adding and removing states is very easy, and only requires you to add/remove a function from an array. The system is always in a defined state, giving less room for unexpected behavior. The code is also very easy to read, making it easy to identify every state and what it does.

{\bfseries Shared resources}

Since enter, loop and leave are different functions, they all have their own separate scope. This causes us to have a few global variables. This is not a big problem, since putting those variable in a namespace makes them local to the file.

\subsection*{\hyperlink{class_stream}{Stream}}

We chose to implement a full duplex stream class, because it gave us a lot of overhead when implementing other modules such as \hyperlink{class_u_a_r_t}{U\+A\+RT}, \hyperlink{namespace_s_p_i}{S\+PI} and \hyperlink{class_c_a_n}{C\+AN}. It utilizes a ring buffer for both input and output. All classes that uses \hyperlink{class_stream}{Stream} inherit from it.

\subsection*{Network}

In addition to \hyperlink{class_c_a_n}{C\+AN}, which is the physical and link layer, we have also made a transport layer and application protocol.

{\bfseries Transport layer -\/ S\+O\+C\+K\+ET}

The transport layer is implemented as a S\+O\+C\+K\+ET (not to be confused with a T\+CP socket). It handles communication between the node and a specified \hyperlink{class_c_a_n}{C\+AN} channel. It uses the stream in order to buffer incoming and outgoing messages.

{\itshape Please see the socket class in the docs for a more detailed description}

{\bfseries Application layer -\/ \hyperlink{class_s_c_p}{S\+CP}}

Our application layer protocol is called \hyperlink{class_s_c_p}{S\+CP} (Simple Command Protocol). It uses a socket to send commands between nodes. The \hyperlink{class_s_c_p}{S\+CP} frame consists of the following\+:


\begin{DoxyItemize}
\item Command, 1 byte.
\item Amount of data, 1 byte
\item Data, 0-\/256 bytes
\end{DoxyItemize}

{\itshape Please see the \hyperlink{class_s_c_p}{S\+CP} class in the docs for a more detailed description}

\subsection*{Extras}

We have implemented a few extra features, listed underneath.

\subsubsection*{Toolchain and C++}

We decided to create our own toolchain, and not use the supplied toolchain in A\+T\+M\+EL Studio. All of our development is done in Linux (except for the G\+AL), and we used \href{http://www.nongnu.org/avrdude/}{\tt A\+V\+R\+D\+U\+DE}, \href{http://www.nongnu.org/avr-libc/}{\tt A\+V\+R-\/\+G++} and \href{https://cmake.org/}{\tt C\+Make} to build and flash the code onto the A\+T\+Mega.

We decided to go for C++ because it makes the code more readable due to modularization.

We used the \href{https://help.ubuntu.com/community/Minicom}{\tt minicom} software to read \hyperlink{class_u_a_r_t}{U\+A\+RT} messages.

\subsubsection*{Extended memory map}

The memory map has been extended from 2\+KB of S\+R\+AM to 8\+KB of S\+R\+AM. We achieved this by not using the J\+T\+AG interface that occupied the remaining address and data pins. This gave us an updated memory map as seen below.



\subsubsection*{\hyperlink{namespace_highscore}{Highscore}, n\+R\+F51 chip and Bluetooth}

We have implemented a highscore system that stores the highscore in the E\+E\+P\+R\+OM at the A\+Tmega162. The highscore system is based on time, and the longer you play, the higher score you get. This is realized through a timer module in the A\+Tmega2560.

When a game ends, you are prompted to enter your name. We have developed an android app, which communicates with a n\+R\+F51 (from now on node 3) dev board through B\+LE (Bluetooth Low Energy). The name is entered on this app and sent to node 3.

When node 3 receives the name, it is sent to node 2 through \hyperlink{namespace_s_p_i}{S\+PI}. Node 2 then sends the name to node 1 through \hyperlink{class_c_a_n}{C\+AN}, which stores the name in the E\+E\+P\+R\+OM (if the highscore is good enough).

This highscore can later be found by the menu entry \char`\"{}\+Highscore\char`\"{}. You can also clear the highscore here through the menu entry clear.

{\bfseries You can find the code for the n\+R\+F51 chip and the android code in the extra folder.}

\subsubsection*{Render \hyperlink{class_o_l_e_d}{O\+L\+ED} from Node 2 over \hyperlink{class_c_a_n}{C\+AN}}

It is possible to render the \hyperlink{class_o_l_e_d}{O\+L\+ED} from node 2 through the \hyperlink{class_c_a_n}{C\+AN} network. However, currently, it does not render fast enough for it to be a viable option, but we have included a menu entry which illustrates that it is possible.

This is done through the \hyperlink{class_s_c_p}{S\+CP} protocol that we developed for this project. Please see the documentation and the \hyperlink{class_s_c_p}{S\+CP} class. We send a \hyperlink{class_c_a_n}{C\+AN} message with \hyperlink{class_s_c_p}{S\+CP} instruction node 1 to write to a specific memory address. The \hyperlink{class_o_l_e_d}{O\+L\+ED} addresses and data is calculated at node 2. 